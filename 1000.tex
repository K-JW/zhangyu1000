\documentclass[openany,twocolumn]{ctexbook}
\usepackage{amsmath,amssymb,mathtools,upgreek}
\usepackage{fourier}
\usepackage{esint}
\usepackage{bm}
\makeatletter
\def\vdots@i#1#2#3{\vbox{
  #1\baselineskip#2\p@ \lineskiplimit\z@
  \kern#3\p@\hbox{.}\hbox{.}\hbox{.}}}
\DeclareRobustCommand\vdots{
  \mathchoice
    {\vdots@i{}{4}{6}}
    {\vdots@i{}{4}{6}}
    {\vdots@i{\scriptsize}{2}{1}}
    {\vdots@i{\tiny}{2}{1}}
}
\makeatother
\usepackage{physics}
\usepackage{siunitx}
\usepackage{lastpage}
\usepackage{graphicx}
\numberwithin{figure}{section}
\renewcommand\thefigure{\arabic{chapter}-\arabic{section}-\arabic{figure}}
\usepackage{floatrow}
\usepackage{subcaption}
% \renewcommand\thesubfigure{(\alph{subfigure})}
% \captionsetup[sub]{labelformat=simple}
\xeCJKDeclareCharClass{FullRight}{"2236}
\newcommand\ratio[2]{#1^^^^2236#2}

\usepackage[a4paper,top=2.5cm,bottom=2.5cm,inner=1.5cm,outer=3cm]{geometry}
% \usepackage[toc]{multitoc}

\usepackage{tikz}
\usetikzlibrary{shapes.geometric,calc}
\newcommand*{\circled}[1]{\lower.7ex\hbox{\tikz\draw (0pt, 0pt)%
	circle (.5em) node {\makebox[1em][c]{\small #1}};}}
\let\libcirc\circled

\makeatletter
\newcommand{\rmnum}[1]{\romannumeral #1}
\newcommand{\Rmnum}[1]{\expandafter\@slowromancap\romannumeral #1@}
\makeatother

\newcommand\score[2]{
\pgfmathsetmacro\pgfxa{#1+1}
\tikzstyle{scorestars}=[star, star points=5, star point ratio=2.25, draw,inner sep=0.15em,anchor=outer point 3]
\begin{tikzpicture}[baseline]
  \foreach \i in {1,...,#2} {
	\pgfmathparse{(\i<=#1?"black":"white")}
	\edef\starcolor{\pgfmathresult}
	\draw (\i*1em,0) node[name=star\i,scorestars,fill=\starcolor]  {};
   }
   \pgfmathparse{(#1>int(#1)?int(#1+1):0}
   \let\partstar=\pgfmathresult
   \ifnum\partstar>0
	 \pgfmathsetmacro\starpart{#1-(int(#1))}
	 \path [clip] ($(star\partstar.outer point 3)!(star\partstar.outer point 2)!(star\partstar.outer point 4)$) rectangle 
	($(star\partstar.outer point 2 |- star\partstar.outer point 1)!\starpart!(star\partstar.outer point 1 -| star\partstar.outer point 5)$);
	 \fill (\partstar*1em,0) node[scorestars,fill=black]  {};
   \fi
,\end{tikzpicture}
}

\usepackage{wallpaper}
\renewcommand{\CenterWallPaper}[2]{%
\AddToShipoutPicture{\put(\LenToUnit{\wpXoffset},\LenToUnit{\wpYoffset}){%
	 \parbox[b][\paperheight]{\paperwidth}{%
		\vfill
		\centering
		\tikz[opacity=0.075] \node[inner sep=0pt] {\includegraphics[angle=90,width=#1\paperwidth,height=#1\paperheight,keepaspectratio]{#2}};%
		\vfill
	 }}
  }
}
% \CenterWallPaper{1}{figure/ctanlion.pdf}
% \CenterWallPaper{1}{figure/wallpaper.pdf}

\usepackage{tabularx,diagbox}

\setlength{\headheight}{13pt}
\makeatletter
\usepackage{fancyhdr}
\pagestyle{fancy}
\fancyhf{}
\fancyhead[LO]{\bfseries \rightmark}
\fancyhead[RE]{\bfseries \leftmark}
% \fancyhead[LE,RO]{
% \@ifundefined{lastpage@lastpage}{%
% 	\score{0}{10}%
% }{%
% 	\score{10 * \thepage / \lastpage@lastpage}{10}%
% }%
% }
\fancyhead[C]{\bfseries \href{https://github.com/sikouhjw/zhangyu1000}{仅供学习使用,严禁商业使用}}
\fancyfoot[C]{\zihao{-5} {\kaishu 不论一个人的数学水平有多高,只要对数学拥有一颗真诚的心,他就在自己的心灵上得到了升华。}---{\itshape SCIbird}}
\fancyhead[LE,RO]{\bfseries --\thepage/\pageref{LastPage}--}
\makeatother

\usepackage{caption}
\captionsetup{labelsep=space}


\usepackage{theorem}
\ctexset{
	chapter={
		name={},
		number=0\arabic{chapter},
	},
	section={
		format={\zihao{4}\bfseries\centering},
		name={第,章},
		aftername={\hspace{1em}},
		number=\chinese{section},
	},
	subsection={
		format={\zihao{-4}\bfseries\raggedright},
		name={,、},
		aftername={\hspace{0bp}},
		number=\chinese{subsection},
	},
	subsubsection={
		format={\zihao{-4}\bfseries\raggedright},
		name={},
		aftername={\hspace{5bp}},
		number={\arabic{section}.\arabic{subsection}.\arabic{subsubsection}},
	},
}

{
	\theoremstyle{change}
	\theoremheaderfont{\bfseries}
	\theorembodyfont{\normalfont}
	\newtheorem{ti}{}[section]
}
\renewcommand{\theti}{\arabic{section}.\arabic{ti}}

{
	\theoremstyle{change}
	\theoremheaderfont{\bfseries}
	\theorembodyfont{\normalfont}
	\newtheorem{titwo}{}[chapter]
}
\renewcommand{\thetitwo}{\arabic{titwo}.}

\newcommand{\hone}[1]{ \uline{\hspace{#1 pc}}}
\def\htwo{\CJKunderline*[hidden = true]{瞻彼阕者虚室生白}}
\def\kuo{ \mbox{(\hspace{1pc})}}

\newcommand{\fourch}[4]{\noindent\begin{tabular}{*{4}{@{}p{1.97cm}}}(\texttt{A})~#1 & (\texttt{B})~#2 & (\texttt{C})~#3 & (\texttt{D})~#4\end{tabular}} % 一行
\newcommand{\twoch}[4]{\noindent\begin{tabular}{*{2}{@{}p{3.94cm}}}(\texttt{A})~#1 & (\texttt{B})~#2\end{tabular}\\\begin{tabular}{*{2}{@{}p{3.94cm}}}(\texttt{C})~#3 & (\texttt{D})~#4\end{tabular}}  %两行
\newcommand{\onech}[4]{\noindent(\texttt{A})~#1 \\ (\texttt{B})~#2 \\ (\texttt{C})~#3 \\ (\texttt{D})~#4}  % 四行

\def\leq{\leqslant}
\def\geq{\geqslant}
\def\ee{\mathrm{e}}
\def\CC{\mathrm{C}}
\def\TT{\mathrm{T}}
\def\AA{\mathrm{A}}
\def\astt{*}
\edef\lim{\lim\limits}
\edef\sum{\sum\limits}
\let\div\relax
\DeclareMathOperator{\div}{div}
\let\grad\relax
\DeclareMathOperator{\grad}{grad}
\DeclareMathOperator{\rot}{rot}
\DeclareMathOperator{\Cov}{Cov}
\long\def\guanggao{
	\vspace*{\fill}
	\begin{center}
		\bfseries 广告位招租
	\end{center}
	\vspace*{\fill}
}
\def\theenumi{\arabic{enumi}}
\def\labelenumi{(\theenumi)}

% \setCJKmainfont{SourceHanSerifCN}[
% UprightFont    = *-Regular,
% BoldFont       = *-Bold,
% ItalicFont     = *-Regular,
% BoldItalicFont = *-Bold
% ]
% \setCJKfamilyfont{kaishu}{FZXKTJW.ttf}
% \def\kaishu{\CJKfamily{kaishu}}

\usepackage[bookmarksopen=true,bookmarksnumbered=true,hidelinks]{hyperref}

\title{\href{https://github.com/sikouhjw/zhangyu1000}{张宇考研数学题源探析经典 1000 题\\(习题分册·数学一)}\thanks{Build time:\today,Releases:}}
\author{张宇}
\date{2019 年 3 月}


\begin{document}
	% \frontmatter
	% \maketitle
	% \input{chapter/chap0.tex}
	% \tableofcontents
	% \mainmatter
	% \input{chapter/chap1.tex}
	% \input{chapter/chap2.tex}
	% \input{chapter/sec11-1.tex}
\input{chapter/sec11-2.tex}
\input{chapter/sec11-3.tex}
\section{数字特征}
	\begin{titwo}
		现有 10 张奖券,其中 8 张为 2 元的,2 张为 5 元的。今从中任取 3 张,则奖金的数学期望为 \kuo.

		\fourch{$6$}{$7.8$}{$9$}{$11.2$}
	\end{titwo}

	\begin{titwo}
		设 $X_{1}$, $X_{2}$, $X_{3}$ 相互独立,且均服从参数为 $\lambda$ 的泊松分布,令 $Y = \frac{1}{3} (X_{1} + X_{2} + X_{3})$,则 $Y^{2}$ 的数学期望为 \kuo.

		\fourch{$\frac{1}{3} \lambda$}{$\lambda^{2}$}{$\frac{1}{3} \lambda + \lambda^{2}$}{$\frac{1}{3} \lambda^{2} + \lambda$}
	\end{titwo}

	\begin{titwo}
		设 $X$ 为连续型随机变量,方差存在,则对任意常数 $C$ 和 $\varepsilon > 0$,必有 \kuo.

		\onech{$P\{ |X - C| \geq \varepsilon \} = E(|X - C|)/\varepsilon$}%
		{$P\{ |X - C| \geq \varepsilon \} \geq E(|X - C|)/\varepsilon$}%
		{$P\{ |X - C| \geq \varepsilon \} \leq E(|X - C|)/\varepsilon$}%
		{$P\{ |X - C| \geq \varepsilon \} \leq DX/\varepsilon^{2}$}
	\end{titwo}

	\begin{titwo}
		一袋中有 6 个正品 4 个次品,按下列方式抽样:每次取 1 个,取后放回,共取 $n(n \leq 10)$ 次,其中次品个数记为 $X$;若一次性取出 $n(n \leq 10)$ 个,其中次品个数记为 $Y$. 则下列正确的是 \kuo.

		\onech{$EX > EY$}{$EX < EY$}{$EX = EY$}{若 $n$ 不同,则 $EX$, $EY$ 大小不同}
	\end{titwo}

	\begin{titwo}
		设随机变量 $(X,Y)$ 的概率密度 $f(x,y)$ 满足 $f(x,$ $y) = f(-x,y)$,且 $\rho_{XY}$ 存在,则 $\rho_{XY} = $ \kuo.

		\fourch{$1$}{$0$}{$-1$}{$-1$ 或 $1$}
	\end{titwo}

	\begin{titwo}
		设随机变量 $(X,Y)$ 服从二维正态分布,其边缘分布为 $X \sim N(1,1)$, $Y \sim N(2,4)$, $X$ 与 $Y$ 的相关系数为 $\rho_{XY} = -\frac{1}{2}$,且概率 $P\{ aX + bY \leq 1 \} = \frac{1}{2}$,则\kuo.

		\twoch{$a = \frac{1}{2}$, $b = - \frac{1}{4}$}{$a = \frac{1}{4}$, $b = - \frac{1}{2}$}{$a = -\frac{1}{4}$, $b = \frac{1}{2}$}{$a = \frac{1}{2}$, $b = \frac{1}{4}$}
	\end{titwo}

	\begin{titwo}
		设 $a$ 为区间 $(0,1)$ 上一个定点,随机变量 $X$ 服从 $(0,1)$ 上的均匀分布. 以 $Y$ 表示点 $X$ 到 $a$ 的距离,当 $X$ 与 $Y$ 不相关时,$a = $ \kuo.

		\fourch{$0.1$}{$0.3$}{$0.5$}{$0.7$}
	\end{titwo}

	\begin{titwo}
		设 $X$ 是随机变量,$EX > 0$ 且 $E \bigl( X^{2} \bigr) = 0.7$, $DX = 0.2$,则以下各式成立的是 \kuo.

		\twoch{$P\bigl\{ - \frac{1}{2} < X < \frac{3}{2} \bigr\} \geq 0.2$}%
		{$P\bigl\{ X > \sqrt{2} \bigr\} \geq 0.6$}%
		{$P\bigl\{ 0 < X < \sqrt{2} \bigr\} \geq 0.6$}%
		{$P\bigl\{0 < X < \sqrt{2}\bigr\} \leq 0.6$}
	\end{titwo}
	
	\begin{titwo}
		设随机变量 $X_{1}$, $X_{2}$, $\cdots$, $X_{n}$ $(n > 1)$ 独立同分布,其方差 $\sigma^{2} > 0$,记 $\overline{X_{k}} = \frac{1}{k} \* \sum_{i=1}^{k} X_{i}$ $(1 \leq k \leq n)$,则 $\Cov\bigl( \overline{X_{s}},\overline{X_{t}} \bigr)$ $(1 \leq s,t \leq n)$ 的值等于 \kuo.

		\twoch{$\frac{\sigma^{2}}{\max\{ s,t \}}$}{$\frac{\sigma^{2}}{\min\{ s,t \}}$}{$\sigma^{2} \cdot \max\{s,t\}$}{$\sigma^{2} \cdot \min\{s,t\}$}
	\end{titwo}

	\begin{titwo}
		设随机变量 $X$ 的概率密度为
		\[
		f(x) = \begin{cases}
			\frac{3}{8}x^{2}, & 0 < x < 2, \\
			0, & \text{其他},
		\end{cases}
		\]
		则 $E\bigl( \frac{1}{X^{2}} \bigr) = $ \htwo.
	\end{titwo}

	\begin{titwo}
		设随机变量 $Y$ 服从参数为 $1$ 的指数分布,记
		\[
			X_{k} = \begin{cases}
				0, & Y \leq k, \\
				1, & Y > k,
			\end{cases} k = 1,2,
		\]
		则 $E(X_{1} + X_{2}) = $ \htwo.
	\end{titwo}

	\begin{titwo}
		已知离散型随机变量 $X$ 服从参数为 $2$ 的泊松分布,即 $P\{ X = k \} = \frac{ 2^{k}\ee^{-2} }{k!}$, $k = 0$, $1$, $2$, $\cdots$,则随机变量 $Z = 3X - 2$ 的数学期望 $EZ = $ \htwo.
	\end{titwo}

	\begin{titwo}
		设随机变量 $X_{1}$, $X_{2}$, $\cdots$, $X_{100}$ 独立同分布,且 $EX_{i} = 0$, $DX_{i} = 10$, $i = 1$, $2$, $\cdots$, $100$,令 $\overline{X} = \frac{1}{100} \* \sum_{i=1}^{100} X_{i}$,则 $E\Biggl[ \sum_{i=1}^{100} \bigl( X_{i} - \overline{X} \bigr)^{2} \Biggr] = $ \htwo.
	\end{titwo}

	\begin{titwo}
		设随机变量 $X$ 和 $Y$ 均服从 $B\bigl( 1,\frac{1}{2} \bigr)$,且 $D(X + Y) = 1$,则 $X$ 与 $Y$ 的相关系数 $\rho = $ \htwo.
	\end{titwo}

	\begin{titwo}
		已知随机变量 $X \sim N(-3,1)$, $Y \sim N(2,1)$,且 $X$, $Y$ 相互独立,设随机变量 $Z = X - 2Y + 7$,则 $Z \sim $ \htwo.
	\end{titwo}

	\begin{titwo}
		设相互独立的两个随机变量 $X$, $Y$ 具有同一分布律,且 $X$ 的分布律为
		\begin{center}
			\begin{tabular}{c|cc}
				\hline
				$X$ & $0$ & $1$ \\
				\hline
				$P$ & $\frac{1}{2}$ & $\frac{1}{2}$ \\
				\hline
			\end{tabular}
		\end{center}
		则随机变量 $Z = \max\{X,Y\}$ 的分布律为 \htwo.
	\end{titwo}

	\begin{titwo}
		设二维随机变量 $(X,Y)$ 的概率密度为
		\[
			f(x,y) = \begin{cases}
				\frac{1}{8} (x + y), & 0 \leq x \leq 2,0 \leq y \leq 2, \\
				0, & \text{其他}.
			\end{cases}
		\]
		则随机变量 $U = X + 2Y$, $V = -X$ 的协方差 $\Cov(U,$ $V) = $ \htwo.
	\end{titwo}

	\begin{titwo}
		一台设备由三个部件构成,在设备运转中各部件需要调整的概率分别为 $0.10$, $0.20$, $0.30$,设备部件状态相互独立,以 $X$ 表示同时需要调整的部件数,则 $X$ 的方差为 \htwo.
	\end{titwo}

	\begin{titwo}
		设 $(X,Y)$ 的概率密度为
		\[
		f(x,y) = \begin{cases}
			1, & 0 \leq |y| \leq x \leq 1, \\
			0, & \text{其他},
		\end{cases}
		\]
		则 $\Cov(X,Y) = $ \htwo.
	\end{titwo}

	\begin{titwo}
		若 $X_{1}$, $X_{2}$, $X_{3}$ 两两不相关,且 $DX_{i} = 1$ $(i = 1,2,$ $3)$,则 $D(X_{1} + X_{2} + X_{3}) = $ \htwo.
	\end{titwo}

	\begin{titwo}
		设随机变量 $X_{1}$, $X_{2}$, $X_{3}$ 相互独立,且 $X_{1} \sim B \bigl( 4,$ $\frac{1}{2} \bigr)$, $X_{2} \sim B \bigl( 6,\frac{1}{3} \bigr)$, $X_{3} \sim B\bigl( 6,\frac{1}{5} \bigr)$,则 $E[ X_{1} \* (X_{1} + X_{2} - X_{3}) ] = $ \htwo.
	\end{titwo}

	\begin{titwo}
		设随机变量 $X$ 与 $Y$ 的分布律为 \begin{tabular}{c|cc}
			\hline
			$X$ & $0$ & $1$ \\
			\hline
			$P$ & $\frac{1}{4}$ & $\frac{3}{4}$ \\
			\hline
		\end{tabular} 与 \begin{tabular}{c|cc}
			\hline
			$Y$ & $0$ & $1$ \\
			\hline
			$P$ & $\frac{1}{2}$ & $\frac{1}{2}$ \\
			\hline
		\end{tabular} 且相关系数 $\rho_{XY} = \frac{\sqrt{3}}{3}$,则 $(X,Y)$ 的分布律为 \htwo.
	\end{titwo}

	\begin{titwo}
		设二维随机变量 $(X,Y)$ 的分布律为
		\begin{center}
			\begin{tabular}{c|ccc}
				\hline
				\diagbox{$X$}{$Y$} & $-1$ & $0$ & $1$ \\
				\hline
				$-5$ & $0$ & $\frac{1}{9}$ & $\frac{1}{3}$ \\
				$-1$ & $\frac{1}{9}$ & $0$ & $\frac{2}{9}$ \\
				$1$ & $\frac{1}{9}$ & $\frac{1}{9}$ & $0$ \\
				\hline
			\end{tabular}
		\end{center}
		则 $X$ 与 $Y$ 的协方差为 \htwo.
	\end{titwo}

	\begin{titwo}
		设二维随机变量 $(X,Y)$ 的概率密度为
		\[
			f(x,y) = \begin{cases}
				3x, & 0 < x < 1,0 < y < x, \\
				0, & \text{其他},
			\end{cases}
		\]
		则随机变量 $Z = X - Y$ 的方差为 \htwo.
	\end{titwo}

	\begin{titwo}
		设总体 $X$ 和 $Y$ 相互独立,且分别服从正态分布 $N(0,4)$ 和 $N(0,7)$, $X_{1}$, $X_{2}$, $\cdots$, $X_{8}$ 和 $Y_{1}$, $Y_{2}$, $\cdots$, $Y_{14}$ 分别来自总体 $X$ 和 $Y$ 的简单随机样本,则统计量 $\bigl| \overline{X} - \overline{Y} \bigr|$ 的数学期望和方差分别为 \htwo.
	\end{titwo}

	\begin{titwo}
		假设一设备在任何长为 $t$ 的时间段内发生故障的次数 $N(t)$ 服从参数为 $\lambda t$ 的泊松分布 $(\lambda > 0)$,设两次故障之间时间间隔为 $T$,则 $ET = $ \htwo.
	\end{titwo}

	\begin{titwo}
		二维正态分布一般表示为 $N \bigl( \mu_{1},\mu_{2};\sigma_{1}^{2},\sigma_{2}^{2};\rho \bigr)$,设 $(X,Y) \sim N(1,1;4,9;0.5)$,令 $Z = 2X - Y$,则 $Z$ 与 $Y$ 的相关系数为 \htwo.
	\end{titwo}

	\begin{titwo}
		设随机变量 $X$ 的概率密度为
		\[
			f(x) = \begin{cases}
				ax, & 0 < x < 2, \\
				cx + b, & 2 \leq x \leq 4, \\
				0, & \text{其他},
			\end{cases}
		\]
		已知 $EX = 2$, $P\{1 < X < 3\} = \frac{3}{4}$,求:
		\begin{enumerate}
			\item $a$, $b$, $c$ 的值;
			\item 随机变量 $Y = \ee^{X}$ 的数学期望和方差.
		\end{enumerate}
	\end{titwo}

	\begin{titwo}
		袋中有 $n$ 张卡片,分别记有号码 $1$, $2$, $\cdots$, $n$,从中有放回地抽取 $k$ 次,每次抽取 $1$ 张,以 $X$ 表示所得号码之和,求 $EX$, $DX$.
	\end{titwo}

	\begin{titwo}
		设随机变量 $U$ 在 $[-2,2]$ 上服从均匀分布,记随机变量
		\[
			X = \begin{cases}
				-1, & U \leq -1, \\
				1, & U > -1,
			\end{cases}
			Y = \begin{cases}
				-1, & U \leq 1, \\
				1, & U > 1,
			\end{cases}
		\]
		求:
		\begin{enumerate}
			\item $\Cov(X,Y)$,并判定 $X$ 与 $Y$ 的独立性;
			\item $D[X(1 + Y)]$.
		\end{enumerate}
	\end{titwo}

	\begin{titwo}
		设试验成功的概率为 $\frac{3}{4}$,失败的概率为 $\frac{1}{4}$,独立重复试验直到成功两次为止,试求试验次数的数学期望.
	\end{titwo}

	\begin{titwo}
		设随机变量服从几何分布,其分布律为
		\[
			P\{ X = k \} = (1 - p)^{k - 1} p, 0 < p < 1, k = 1,2,\cdots,
		\]
		求 $EX$ 与 $DX$.
	\end{titwo}

	\begin{titwo}
		设 $(X,Y)$ 的概率密度为
		\[
			f(x,y) = \begin{cases}
				4xy \ee^{-(x^{2} + y^{2})}, & x > 0, y > 0, \\
				0, & \text{其他},
			\end{cases}
		\]
		求 $Z = \sqrt{X^{2} + Y^{2}}$ 的数学期望.
	\end{titwo}

	\begin{titwo}
		设连续型随机变量 $X$ 的所有可能取值在区间 $[a,b]$ 之内,证明:
		\begin{enumerate}
			\item $a \leq EX \leq b$;
			\item $DX \leq \frac{(b-a)^{2}}{4}$.
		\end{enumerate}
	\end{titwo}

	\begin{titwo}
		$\triangle ABC$ 边 $AB$ 上的高 $CD$ 长度为 $h$. 向 $\triangle ABC$ 中随机投掷一点 $P$,求
		\begin{enumerate}
			\item 点 $P$ 到边 $AB$ 的距离 $X$ 的概率密度;
			\item $X$ 的期望与方差.
		\end{enumerate}
	\end{titwo}

	\begin{titwo}
		对三台仪器进行检验,各台仪器产生故障的概率分别为 $p_{1}$, $p_{2}$, $p_{3}$,各台仪器是否产生故障相互独立,求产生故障仪器的台数 $X$ 的数学期望和方差.
	\end{titwo}

	\begin{titwo}
		一商店经销某种商品,每周进货量 $X$ 与顾客对该种商品的需求量 $Y$ 是相互独立的随机变量,且都服从区间 $[10,20]$ 上的均匀分布. 商店每售出一单位商品可得利润 $1000$ 元; 若需求量超过了进货量,商店可从其他商店调剂供应,这时每单位商品可得利润 $500$ 元,试计算此商店经销该种商品每周所得利润的期望值.
	\end{titwo}

	\begin{titwo}
		设 $X$, $Y$, $Z$ 是三个两两不相关的随机变量,数学期望全为零,方差都是 $1$,求 $X - Y$ 和 $Y - Z$ 的相关系数.
	\end{titwo}

	\begin{titwo}
		设二维随机变量 $(X,Y)$ 的概率密度为
		\[
			f(x,y) = \begin{cases}
				2 - x - y, & 0 < x < 1,0 < y < 1, \\
				0, & \text{其他},
			\end{cases}
		\]
		求:
		\begin{enumerate}
			\item 方差 $D(XY)$;
			\item 协方差 $\Cov(3X + Y, X - 2Y)$.
		\end{enumerate}
	\end{titwo}

	\begin{titwo}
		设 $X$ 的概率密度为
		\[
			f(x) = \begin{cases}
				\frac{1}{4}, & |x| < 1, \\
				\frac{1}{8}, & 1 \leq |x| \leq 3, \\
				0, & \text{其他}.
			\end{cases}
		\]
		令
		\[
			Y = g(X) = \begin{cases}
				X^{2} + 1, & X < 1, \\
				2, & X \geq 1,
			\end{cases}
		\]
		求:
		\begin{enumerate}
			\item $F_{Y}(y)$;
			\item $\Cov(X,Y)$.
		\end{enumerate}
	\end{titwo}

	\begin{titwo}
		设二维随机变量 $(U,V)$ 的概率密度为
		\[
			f(u,v) = \begin{cases}
				1, & 0 < u < 1, 0 < v < 2u, \\
				0, & \text{其他}.
			\end{cases}
		\]
		又设 $X$ 与 $Y$ 都是离散型随机变量,其中 $X$ 只取 $-1$, $0$, $1$ 三个值,$Y$ 只取 $-1$, $1$ 两个值,且 $EX = 0.2$, $EY = 0.4$. 又
		\begin{align*}
			P\{X = -1,Y = 1\} &= P\{X = 1,Y = -1\} = P\{X = 0,Y = 1\}\\
			&= \frac{1}{3} P\Biggl\{ V \leq \frac{1}{2} \Biggl| U \leq \frac{1}{2} \Biggr\}.
		\end{align*}
		求:
		\begin{enumerate}
			\item $(X,Y)$ 的概率分布;
			\item $\Cov(X,Y)$.
		\end{enumerate}
	\end{titwo}

	\begin{titwo}
		产品寿命 $X$ 是一个随机变量,其分布函数与概率密度分别为 $F(x)$, $f(x)$. 产品已工作到时刻 $x$,在时刻 $x$ 后的单位时间 $\Delta x$ 内发生失效的概率称为产品在时刻 $x$ 的瞬时失效率,记为 $\lambda(x)$.
		\begin{enumerate}
			\item 证明 $\lambda(x) = \frac{f(x)}{1 - F(x)}$;
			\item 设某产品寿命的瞬时失效率函数为 $\lambda(x) = \alpha$,其中参数 $\alpha > 0$,求产品寿命 $X$ 的数学期望.
		\end{enumerate}
	\end{titwo}

	\begin{titwo}
		把一颗骰子独立地投掷 $n$ 次,记 $1$ 点出现的次数为随机变量 $X$,$6$ 点出现的次数为随机变量 $Y$,记
		\begin{align*}
		X_{i} &= \begin{cases}
			1, & \text{第 $i$ 次投掷出现 $1$ 点}, \\
			0, & \text{其他},
		\end{cases} \\
		Y_{j} &= \begin{cases}
			1, & \text{第 $j$ 次投掷出现 $6$ 点}, \\
			0, & \text{其他},
		\end{cases}
		\end{align*}
		$i$, $j = 1$, $2$, $\cdots$, $n$.
		\begin{enumerate}
			\item 求 $EX$, $DX$;
			\item 分别求 $i \ne j$ 时与 $i = j$ 时 $E(X_{i} Y_{j})$ 的值;
			\item 求 $X$ 与 $Y$ 的相关系数.
		\end{enumerate}
	\end{titwo}

	\begin{titwo}
		商店销售某种季节性商品,每售出一件获利 $500$ 元,季度末未售出的商品每件亏损 $100$ 元,以 $X$ 表示该季节此种商品的需求量,若 $X$ 服从正态分布 $N(100,4)$,问:
		\begin{enumerate}
			\item 进货量最少为多少时才能以超过 \SI{95}{\percent} 的概率保证供应?
			\item 进货量为多少时商店获利的期望值最大?
		\end{enumerate}
		($\varPhi(1.65) = 0.95$, $\varPhi(0.95) = 0.83$,其中 $\varPhi(x)$ 为标准正态分布函数)
	\end{titwo}

	\begin{titwo}
		设 $X$ 和 $Y$ 相互独立且均服从 0-1 分布,$P\{X = 1\} = P\{Y = 1\} = 0.6$. 试证明:$U = X + Y$, $V = X - Y$ 不相关且不独立.
	\end{titwo}

	\begin{titwo}
		对于任意两个事件 $A_{1}$, $A_{2}$,考虑随机变量
		\[
			X_{i} = \begin{cases}
				1, & \text{若事件 $A_{i}$ 出现}, \\
				0, & \text{若事件 $A_{i}$ 不出现}
			\end{cases}(i = 1,2).
		\]
		试证:随机变量 $X_{1}$ 和 $X_{2}$ 独立的充分必要条件是事件 $A_{1}$ 和 $A_{2}$ 相互独立.
	\end{titwo}

	\begin{titwo}
		某商品一周的需求量 $X$ 是随机变量,已知其概率密度为 $f(x) = \begin{cases}
			x\ee^{-x}, & x > 0, \\
			0, & \text{其他}.
		\end{cases}$ 假设各周的需求量相互独立,以 $U_{k}$ 表示 $k$ 周的总需求量,试求:
		\begin{enumerate}
			\item $U_{2}$ 和 $U_{3}$ 的概率密度 $f_{k}(x)$ $(k = 2,3)$;
			\item 接连三周中的周最大需求量的概率密度 $f_{(3)}(x)$.
		\end{enumerate}
	\end{titwo}

	\begin{titwo}
		假设 $G = \bigl\{ (x,y) | x^{2} + y^{2} \leq r^{2} \bigr\}$,而随机变量 $X$ 和 $Y$ 的联合分布是在区域 $G$ 上的均匀分布. 试确定随机变量 $X$ 和 $Y$ 的独立性和相关性.
	\end{titwo}
\section{大数定律与中心极限定理}
	\begin{titwo}
		已知随机变量 $X_{n}(n = 1,2,\cdots)$ 相互独立且都在 $(-1,1)$ 上服从均匀分布,根据独立同分布中心极限定理有 $\lim_{n \to \infty} P \biggl\{ \sum_{i=1}^{n} X_{i} \leq \sqrt{n} \biggr\} = $ \kuo.

		\fourch{$\varPhi(0)$}{$\varPhi(1)$}{$\varPhi\bigl(\sqrt{3}\bigr)$}{$\varPhi(2)$}
	\end{titwo}

	\begin{titwo}
		设随机变量 $X$ 的数学期望 $EX = 75$,方差 $DX = 5$,由切比雪夫不等式估计得
		\[
			P\{ |X - 75| \geq k \} \leq 0.05,
		\]
		则 $k = $ \htwo.
	\end{titwo}

	\begin{titwo}
		设 $X_{1}$, $X_{2}$, $\cdots$, $X_{n}$ 是相互独立的随机变量序列,且都服从参数为 $\lambda$ 的泊松分布,则 $\lim_{n \to \infty} P \Biggl\{ \frac{ \sum_{i=1}^{n} X_{i} - n \lambda }{\sqrt{n \lambda}} \leq x \Biggr\} = $ \htwo.
	\end{titwo}

	\begin{titwo}
		设 $Y \sim \chi^{2}(200)$,则由中心极限定理得 $P\{ Y \leq 200 \}$ 近似等于 \htwo.
	\end{titwo}

	\begin{titwo}
		设 $\{ X_{n} \}$ 是一随机变量序列,$X_{n}$ $(n = 1$, $2$, $\cdots)$ 的概率密度为
		\[
			f_{n}(x) = \frac{n}{\uppi( 1 + n^{2} x^{2} )}, -\infty < x < +\infty,
		\]
		试证:$X_{n}  \xrightarrow{P} 0$.
	\end{titwo}

	\begin{titwo}
		设 $X_{1}$, $X_{2}$, $\cdots$, $X_{n}$ 是独立同分布的随机变量序列,$EX_{i} = \mu$, $DX_{i} = \sigma^{2}$, $i = 1$, $2$, $\cdots$, $n$,令 $Y_{n} = \frac{2}{n(n + 1)} \* \sum_{i=1}^{n} iX_{i}$. 证明:随机变量序列 $\{Y_{n}\}$ 依概率收敛于 $\mu$.
	\end{titwo}

	\begin{titwo}
		若 $DX = 0.004$,利用切比雪夫不等式估计概率 $P\{ |X - EX| < 0.2 \}$.
	\end{titwo}

	\begin{titwo}
		用切比雪夫不等式确定,掷一均质硬币时,需掷多少次,才能保证“正面”出现的频率在 $0.4$ 至 $0.6$ 之间的概率不小于 $0.9$.
	\end{titwo}

	\begin{titwo}
		设事件 $A$ 出现的概率为 $p = 0.5$,试利用切比雪夫不等式,估计在 $1000$ 次独立重复试验中事件 $A$ 出现的次数在 $450$ 到 $550$ 次之间的概率 $\alpha$.
	\end{titwo}

	\begin{titwo}
		设随机变量 $X$ 的概率密度为 $f(x)$,已知方差 $DX = 1$,而随机变量 $Y$ 的概率密度为 $f(-y)$,且 $X$ 与 $Y$ 的相关系数为 $-\frac{1}{4}$. 记 $Z = X + Y$.
		\begin{enumerate}
			\item 求 $EZ$, $DZ$;
			\item 用切比雪夫不等式估计 $P\{|Z| \geq 2\}$.
		\end{enumerate}
	\end{titwo}

	\begin{titwo}
		某计算机系统有 $100$ 个终端,每个终端有 \SI{20}{\percent} 的时间在使用,若各个终端使用与否相互独立,试求有 $10$ 个或更多个终端在使用的概率.
	\end{titwo}

	\begin{titwo}
		某保险公司接受了 \num{10000} 辆电动自行车的保险,每辆车每年的保费为 $12$ 元. 若车丢失,则赔偿车主 $1000$ 元. 假设车的丢失率为 $0.006$,对于此项业务,试利用中心极限定理,求保险公司:
		\begin{enumerate}
			\item 亏损的概率 $\alpha$;
			\item 一年获利润不少于 \num{40000} 元的概率 $\beta$;
			\item 一年获利润不少于 \num{60000} 元的概率 $\gamma$.
		\end{enumerate}
	\end{titwo}

	\begin{titwo}
		一生产线生产的产品成箱包装,每箱的重量是随机的,假设每箱平均重量 $50$ 千克,标准差为 $5$ 千克,若用最大载重为 $5$ 吨的汽车承运,试用中心极限定理说明每辆车最多可装多少箱,才能保证不超载的概率大于 $0.977$ $(\varPhi(2) = 0.977)$.
	\end{titwo}
\section{统计量}
	\begin{titwo}
		设 $X_{1}$, $X_{2}$, $\cdots$, $X_{n}$ $(n > 1)$ 是来自总体 $N(0,1)$ 的简单随机样本,记 $\overline{X} = \frac{1}{n} \sum_{i=1}^{n} X_{i}$, $Q^{2} = \sum_{i=1}^{n} X_{i}^{2}$,则 \kuo.

		\onech{$\overline{X} \sim N(0,1)$, $Q^{2} \sim \chi^{2}(n)$}%
		{$\overline{X} \sim N(0,n)$, $Q^{2} \sim \chi^{2}(n-1)$}%
		{$\overline{X} \sim N\bigl(0,\frac{1}{n}\bigr)$, $Q^{2} \sim \chi^{2}(n)$}%
		{$\overline{X} \sim N\bigl(0,\frac{1}{n}\bigr)$, $Q^{2} \sim \chi^{2}(n-1)$}
	\end{titwo}

	\begin{titwo}
		设 $X_{1}$, $X_{2}$, $\cdots$, $X_{8}$ 是来自总体 $N(2,1)$ 的简单随机样本,则统计量
		\[
			Y = \frac{2(X_{1} + X_{2} + X_{3} - 6)}{\sqrt{ 3(X_{4} + X_{5} - 4)^{2} + 2(X_{6} + X_{7} + X_{8} - 6)^{2} }}
		\]
		服从 \kuo.

		\fourch{$\chi^{2}(2)$}{$\chi^{2}(3)$}{$t(2)$}{$t(3)$}
	\end{titwo}

	\begin{titwo}
		设 $X_{1}$, $X_{2}$, $\cdots$, $X_{n}$ 是来自总体 $X \sim N(0,1)$ 的简单随机样本,则统计量
		\[
			Y = \frac{ \sqrt{n - 1}X_{1} }{\sqrt{ \sum_{i=2}^{n} X_{i}^{2} }}
		\]
		服从 \kuo.

		\twoch{$Y \sim \chi^{2}(n-1)$}{$Y \sim t(n-1)$}{$Y \sim F(n,1)$}{$Y \sim F(1,n-1)$}
	\end{titwo}

	\begin{titwo}
		设随机变量 $X \sim F(n,n)$,记 $p_{1} = P\{X \geq 1\}$, $p_{2} = P\{X \leq 1\}$,则 \kuo.

		\twoch{$p_{1} < p_{2}$}{$p_{1} > p_{2}$}{$p_{1} = p_{2}$}{$p_{1}$, $p_{2}$ 大小无法比较}
	\end{titwo}

	\begin{titwo}
		设总体 $X \sim N \bigl( a,\sigma^{2} \bigr)$, $Y \sim N\bigl( b,\sigma^{2} \bigr)$,且相互独立. 分别从 $X$ 和 $Y$ 中各抽取容量为 $9$ 和 $10$ 的简单随机样本,记它们的方差为 $S_{X}^{2}$ 和 $S_{Y}^{2}$,并记 $S_{12}^{2} = \frac{1}{2} \* \bigl( S_{X}^{2} + S_{Y}^{2} \bigr)$ 和 $S_{XY}^{2} = \frac{1}{18} \bigl( 8 S_{X}^{2} + 10 S_{Y}^{2} \bigr)$,则这四个统计量 $S_{X}^{2}$, $S_{Y}^{2}$, $S_{12}^{2}$, $S_{XY}^{2}$ 中,方差最小者是 \kuo.

		\fourch{$S_{X}^{2}$}{$S_{Y}^{2}$}{$S_{12}^{2}$}{$S_{XY}^{2}$}
	\end{titwo}

	\begin{titwo}
		设 $X_{1}$, $X_{2}$, $\cdots$, $X_{n}$ 是取自总体 $N\bigl( \mu,\sigma^{2} \bigr)$ 的样本,$\overline{X}$ 是样本均值,记
		\begin{gather*}
			S_{1}^{2} = \frac{1}{n-1} \sum_{i=1}^{n} \bigl( X_{i} - \overline{X} \bigr)^{2}, S_{2}^{2} = \frac{1}{n} \sum_{i=1}^{n} \bigl( X_{i} - \overline{X} \bigr)^{2}, \\
			S_{3}^{2} = \frac{1}{n-1} \sum_{i=1}^{n} (X_{i} - \mu)^{2}, S_{4}^{2} = \frac{1}{n} \sum_{i=1}^{n} (X_{i} - \mu)^{2},
		\end{gather*}
		则服从自由度为 $n-1$ 的 $t$ 分布的随机变量是 \kuo.

		\fourch{$\frac{\overline{X} - \mu}{S_{1}/\sqrt{n-1}}$}{$\frac{\overline{X} - \mu}{S_{2}/\sqrt{n-1}}$}{$\frac{\overline{X} - \mu}{S_{3}/\sqrt{n}}$}{$\frac{\overline{X} - \mu}{S_{4}/\sqrt{n}}$}
	\end{titwo}

	\begin{titwo}
		设总体 $X$ 服从正态分布 $N\bigl( \mu,\sigma^{2} \bigr)$, $X_{1}$, $X_{2}$, $\cdots$, $X_{n}$ 是取自总体的简单随机样本,样本均值为 $\overline{X}$,样本方差为 $S^{2}$,则服从 $\chi^{2}(n)$ 的随机变量为 \kuo.

		\twoch{$\frac{\overline{X}^{2}}{\sigma^{2}} + \frac{(n-1)S^{2}}{\sigma^{2}}$}{$\frac{n\overline{X}^{2}}{\sigma^{2}} + \frac{(n-1)S^{2}}{\sigma^{2}}$}{$\frac{(\overline{X} - \mu)^{2}}{\sigma^{2}} + \frac{(n-1)S^{2}}{\sigma^{2}}$}{$\frac{n(\overline{X} - \mu)^{2}}{\sigma^{2}} + \frac{(n-1)S^{2}}{\sigma^{2}}$}
	\end{titwo}

	\begin{titwo}
		设总体 $X$ 与 $Y$ 都服从正态分布 $N\bigl(0,\sigma^{2}\bigr)$,已知 $X_{1}$, $X_{2}$, $\cdots$, $X_{m}$ 与 $Y_{1}$, $Y_{2}$, $\cdots$, $Y_{n}$ 是分别来自总体 $X$ 与 $Y$ 的两个相互独立的简单随机样本,统计量 $Y = \frac{ 2(X_{1} + X_{2} + \cdots + X_{m}) }{ \sqrt{Y_{1}^{2} + Y_{2}^{2} + \cdots + Y_{n}^{2}} }$ 服从 $t(n)$ 分布,则 $\frac{m}{n} = $ \kuo.

		\fourch{$1$}{$\frac{1}{2}$}{$\frac{1}{3}$}{$\frac{1}{4}$}
	\end{titwo}

	\begin{titwo}
		设总体 $X$ 服从正态分布 $N\bigl( \mu,\sigma^{2} \bigr)$, $X_{1}$, $X_{2}$, $\cdots$, $X_{n}$ $(n > 1)$ 是取自总体的简单随机样本,样本均值为 $\overline{X}$,如果 $P \{ |X - \mu| < a \} = P \bigl\{ \bigl| \overline{X} - \mu \bigr| < b \bigr\}$,则比值 $\frac{a}{b}$ \kuo.

		\onech{与 $\sigma$ 及 $n$ 都有关}{与 $\sigma$ 及 $n$ 都无关}{与 $\sigma$ 无关,与 $n$ 有关}{与 $\sigma$ 有关,与 $n$ 无关}
	\end{titwo}

	\begin{titwo}
		设 $X_{1}$, $X_{2}$, $\cdots$, $X_{8}$ 和 $Y_{1}$, $Y_{2}$, $\cdots$, $Y_{10}$ 分别是来自正态总体 $N(-1,4)$ 和 $N(2,5)$ 的简单随机样本,且相互独立,$S_{1}^{2}$, $S_{2}^{2}$ 分别为这两个样本的方差,则服从 $F(7,9)$ 分布的统计量是 \kuo.

		\fourch{$\frac{2 S_{1}^{2}}{5 S_{2}^{2}}$}{$\frac{4 S_{2}^{2}}{5 S_{1}^{2}}$}{$\frac{5 S_{1}^{2}}{2 S_{2}^{2}}$}{$\frac{5 S_{1}^{2}}{4 S_{2}^{2}}$}
	\end{titwo}

	\begin{titwo}
		设 $X_{1}$, $X_{2}$, $\cdots$, $X_{n}$ 独立同分布 $N\bigl( \mu,\sigma^{2} \bigr)$,令 $\overline{X} = \frac{1}{n} \sum_{i=1}^{n} X_{i}$, $V_{i} = X_{i} - \overline{X}$, $i = 1$, $2$, $\cdots$, $n$,则
		\[
			Z_{k} = \frac{(k+1)V_{k} + V_{k+1} + \cdots + V_{n-1}}{\sigma\sqrt{k(k+1)}}
		\]
		$(k = 1,2,\cdots,n-1)$ 服从的分布为 \kuo.

		\fourch{$t(n-1)$}{$N(0,1)$}{$\chi^{2}(1)$}{$F(1,1)$}
	\end{titwo}

	\begin{titwo}
		设总体 $X \sim P(\lambda)$, $X_{1}$, $X_{2}$, $\cdots$, $X_{n}$ 是来自总体 $X$ 的简单随机样本,它的均值和方差分别为 $\overline{X}$ 和 $S^{2}$,则 $E\Bigl( \overline{X}^{2} \Bigr)$ 和 $E\bigl( S^{2} \bigr)$ 分别为 \htwo.
	\end{titwo}

	\begin{titwo}
		设 $X_{1}$, $X_{2}$, $X_{3}$, $X_{4}$ 是来自正态总体 $X \sim N \bigl( \mu,\sigma^{2} \bigr)$ 的简单随机样本,则统计量
		\[
			Y = \frac{X_{3} - X_{4}}{\sqrt{ (X_{1} - \mu)^{2} + (X_{2} - \mu)^{2} }}
		\]
		服从的分布是 \htwo.
	\end{titwo}

	\begin{titwo}
		设总体 $X \sim N(a,2)$, $Y \sim N(b,2)$,且独立,由分别来自总体 $X$ 和 $Y$ 的容量分别为 $m$ 和 $n$ 的简单随机样本得样本方差 $S_{X}^{2}$ 和 $S_{Y}^{2}$,则统计量 $T = \frac{1}{2} \bigl[ (m-1) \* S_{X}^{2} + (n-1) \* S_{Y}^{2} \bigr]$ 服从的分布是 \htwo.
	\end{titwo}

	\begin{titwo}
		设 $X_{1}$, $X_{2}$, $\cdots$, $X_{n}$ 为来自总体 $X$ 的一个简单随机样本,$EX = \mu$, $DX = \sigma^{2} < +\infty$,求 $E\overline{X}$, $D\overline{X}$ 和 $E \bigl( S^{2} \bigr)$.
	\end{titwo}

	\begin{titwo}
		从装有 $1$ 个白球和 $2$ 个黑球的罐子里有放回地取球,记
		\[
			X = \begin{cases}
				0, & \text{取到白球}, \\
				1, & \text{取到黑球},
			\end{cases}
		\]
		这样连续取 $5$ 次得样本 $X_{1}$, $X_{2}$, $X_{3}$, $X_{4}$, $X_{5}$. 记 $Y = X_{1} + X_{2} + \cdots + X_{5}$,求:
		\begin{enumerate}
			\item $Y$ 的分布律,$EY$, $E\bigl( Y^{2} \bigr)$;
			\item $E\overline{X}$, $E\bigl( S^{2} \bigr)$ (其中 $\overline{X}$, $S^{2}$ 分别为样本 $X_{1}$, $X_{2}$, $\cdots$, $X_{5}$ 的均值与方差).
		\end{enumerate}
	\end{titwo}
\section{点估计}
	\begin{titwo}
		设 $x_{1}$, $x_{2}$, $\cdots$, $x_{n}$ 是来自总体 $X \sim N\bigl( \mu,\sigma^{2} \bigr)$ ($\mu$, $\sigma^{2}$ 都未知) 的简单随机样本的观察值,则 $\sigma^{2}$ 的最大似然估计值为 \kuo.

		\twoch{$\frac{1}{n} \sum_{i=1}^{n} (x_{i} - \mu)^{2}$}{$\frac{1}{n} \sum_{i=1}^{n} \bigl(x_{i} - \overline{x}\bigr)^{2}$}{$\frac{1}{n-1} \sum_{i=1}^{n} (x_{i} - \mu)^{2}$}{$\frac{1}{n-1} \sum_{i=1}^{n} \bigl(x_{i} - \overline{x}\bigr)^{2}$}
	\end{titwo}

	\begin{titwo}
		设 $X \sim P(\lambda)$,其中 $\lambda > 0$ 是未知参数,$x_{1}$, $x_{2}$, $\cdots$, $x_{n}$ 是总体 $X$ 的一组样本值,则 $P\{X = 0\}$ 的最大似然估计值为 \kuo.

		\twoch{$\ee^{-\frac{1}{\overline{x}}}$}{$\frac{1}{n} \sum_{i=1}^{n} \ln x_{i}$}{$\frac{1}{\ln \overline{x}}$}{$\ee^{-\overline{x}}$}
	\end{titwo}

	\begin{titwo}
		设总体 $X \sim P(\lambda)$ ($\lambda$ 为未知参数),$X_{1}$, $X_{2}$, $\cdots$, $X_{n}$ 是来自总体 $X$ 的简单随机样本,其均值与方差分别为 $\overline{X}$ 与 $S^{2}$,则为使 $\hat \lambda = a \* \overline{X} + (2-3a) \* S^{2}$ 是 $\lambda$ 的无偏估计量,常数 $a$ 应为 \kuo.

		\fourch{$-1$}{$0$}{$\frac{1}{2}$}{$1$}
	\end{titwo}

	\begin{titwo}
		设总体 $X$ 的概率密度为
		\[
			f(x) = \begin{cases}
				(\theta + 1) x^{\theta}, & 0 < x < 1, \\
				0, & \text{其他},
			\end{cases}
		\]
		其中 $\theta > -1$ 为参数. $x_{1}$, $x_{2}$, $\cdots$, $x_{n}$ 是来自总体 $X$ 的样本观测值,则未知参数 $\theta$ 的最大似然估计值为 \htwo.
	\end{titwo}

	\begin{titwo}
		设总体 $X$ 的概率密度为
		\[
		f(x;\theta) = \begin{cases}
			\sqrt{\theta} x^{ \sqrt{\theta} - 1 }, & 0 < x < 1, \\
			0, & \text{其他},
		\end{cases}
		\]
		其中 $\theta > 0$ 为未知参数,又设 $x_{1}$, $x_{2}$, $\cdots$, $x_{n}$ 是 $X$ 的一组样本值,则参数 $\theta$ 的最大似然估计值为 \htwo.
	\end{titwo}

	\begin{titwo}
		设总体 $X$ 的概率密度为
		\[
			f(x;\theta) = \begin{cases}
				\frac{6x}{\theta^{3}} (\theta - x), & 0 < x < \theta, \\
				0, & \text{其他},
			\end{cases}
		\]
		又设 $X_{1}$, $X_{2}$, $\cdots$, $X_{n}$ 是来自 $X$ 的一个简单随机样本,求未知参数 $\theta$ 的矩估计量 $\hat \theta$,并求 $E \hat \theta$ 和 $D \hat \theta$.
	\end{titwo}

	\begin{titwo}
		设总体 $X$ 的概率密度为
		\[
			f(x;\alpha) = \begin{cases}
				(\alpha + 1) x^{\alpha}, & 0 < x < 1, \\
				0, & \text{其他},
			\end{cases}
		\]
		试用样本 $X_{1}$, $X_{2}$, $\cdots$, $X_{n}$ 求参数 $\alpha$ 的矩估计和最大似然估计.
	\end{titwo}

	\begin{titwo}
		设 $X_{1}$, $X_{2}$, $\cdots$, $X_{n}$ 是来自对数级数分布
		\[
			P\{X = k\} = - \frac{1}{\ln(1 - p)} \frac{p^{k}}{k} (0 < p < 1, k = 1,2,\cdots)
		\]
		的一个样本,求 $p$ 的矩估计.
	\end{titwo}

	\begin{titwo}
		设总体 $X$ 服从参数为 $N$ 和 $p$ 的二项分布,$X_{1}$, $X_{2}$, $\cdots$, $X_{n}$ 为取自 $X$ 的简单随机样本,试求参数 $N$ 和 $p$ 的矩估计.
	\end{titwo}

	\begin{titwo}
		设总体 $X$ 的分布列为截尾几何分布
		\begin{gather*}
			P\{X = k\} = \theta^{k-1} (1 - \theta), k = 1,2,\cdots,r, \\
			P\{X = r + 1\} = \theta^{r},
		\end{gather*}
		从中抽得样本 $X_{1}$, $X_{2}$, $\cdots$, $X_{n}$,其中有 $m$ 个取值为 $r+1$,求 $\theta$ 的最大似然估计.
	\end{titwo}

	\begin{titwo}
		设 $X_{1}$, $X_{2}$, $\cdots$, $X_{n}$ 是取自均匀分布 $(0,\theta)$ 上的一个样本,试证:$T_{n} = \max\{ X_{1}$, $X_{2}$, $\cdots$, $X_{n} \}$ 是 $\theta$ 的相合估计量.
	\end{titwo}

	\begin{titwo}
		设 $X_{1}$, $X_{2}$, $\cdots$, $X_{n}$ 为来自总体 $X$ 的简单随机样本,且 $X$ 的概率密度为
		\[
			f(x) = \begin{cases}
				\frac{4x^{2}}{\alpha^{3} \sqrt{\uppi}} \ee^{ - (\frac{x}{\alpha})^{2} }, & x > 0, \alpha > 0, \\
				0, & \text{其他}.
			\end{cases}
		\]
		\begin{enumerate}
			\item 求未知参数 $\alpha$ 的矩估计和最大似然估计;
			\item 验证所求得的矩估计是否为 $\alpha$ 的无偏估计.
		\end{enumerate}
	\end{titwo}

	\begin{titwo}
		设从均值为 $\mu$,方差为 $\sigma^{2} > 0$ 的总体中分别抽取容量为 $n_{1}$, $n_{2}$ 的两个独立样本,样本均值分别为 $\overline{X}$, $\overline{Y}$. 证明:对于任何满足条件 $a + b = 1$ 的常数 $a$, $b$, $T = a \* \overline{X} + b \* \overline{Y}$ 是 $\mu$ 的无偏估计量,并确定常数 $a$, $b$ 的值,使得方差 $DT$ 达到最小.
	\end{titwo}

	\begin{titwo}
		设总体 $X \sim N \bigl( \mu_{1},\sigma^{2} \bigr)$, $Y \sim N\bigl( \mu_{2},\sigma^{2} \bigr)$. 从总体 $X$, $Y$ 中独立地抽取两个容量为 $m$, $n$ 的样本 $X_{1}$, $X_{2}$, $\cdots$, $X_{m}$ 和 $Y_{1}$, $Y_{2}$, $\cdots$, $Y_{n}$. 记样本均值分别为 $\overline{X}$, $\overline{Y}$. 若 $Z = C \Bigl[ \bigl( \overline{X} - \mu_{1} \bigr)^{2} + \bigl( \overline{Y} - \mu_{2} \bigr)^{2} \Bigr]$ 是 $\sigma^{2}$ 的无偏估计. 求:
		\begin{enumerate}
			\item $C$;
			\item $Z$ 的方差 $DZ$.
		\end{enumerate}
	\end{titwo}

	\begin{titwo}
		设 $X_{1}$, $X_{2}$, $\cdots$, $X_{n}$ 为来自总体 $X$ 的简单随机样本,且 $X$ 的概率分布为
		\[
			X \sim \begin{psmallmatrix}
				1 & 2 & 3 \\
				\theta^{2} & 2 \theta (1 - \theta) & (1 - \theta)^{2}
			\end{psmallmatrix},
		\]
		其中 $0 < \theta < 1$. 分别以 $v_{1}$, $v_{2}$ 表示 $X_{1}$, $X_{2}$, $\cdots$, $X_{n}$ 中 $1$, $2$ 出现的次数,试求:
		\begin{enumerate}
			\item 未知参数 $\theta$ 的最大似然估计量;
			\item 未知参数 $\theta$ 的矩估计量;
			\item 当样本值为 $1$, $1$, $2$, $1$, $3$, $2$ 时的最大似然估计值和矩估计值.
		\end{enumerate}
	\end{titwo}

	\begin{titwo}
		假设一批产品的不合格品数与合格品数之比为 $R$ (未知常数). 现在按还原抽样方式随意抽取的 $n$ 件中发现 $k$ 件不合格品. 试求 $R$ 的最大似然估计值.
	\end{titwo}

	\begin{titwo}
		设袋中有编号为 $1 \sim N$ 的 $N$ 张卡片,其中 $N$ 未知,现从中有放回地任取 $n$ 张,所得号码为 $x_{1}$, $x_{2}$, $\cdots$, $x_{n}$.
		\begin{enumerate}
			\item 求 $N$ 的矩估计量 $\hat N_{1}$,并计算概率 $P\bigl\{ \hat N_{1} = 1 \bigr\}$;
			\item 求 $N$ 的最大似然估计量 $\hat N_{2}$,并求 $\hat N_{2}$ 的分布律.
		\end{enumerate}
	\end{titwo}

	\begin{titwo}
		设 $X_{1}$, $X_{2}$, $\cdots$, $X_{n}$ 是来自总体 $X$ 的简单随机样本,$X$ 的概率密度为
		\[
			f(x;\theta) = \begin{cases}
				\frac{x}{\theta^{2}} \ee^{ - \frac{x^{2}}{2 \theta^{2}} }, & x > 0, \\
				0, & x \leq 0,
			\end{cases}
		\]
		其中 $\theta > 0$,试求 $\theta$ 的最大似然估计.
	\end{titwo}

	\begin{titwo}
		设 $X_{1}$, $X_{2}$, $\cdots$, $X_{n}$ 是来自总体 $X$ 的一个简单随机样本,$\hat \theta_{n}(X_{1},X_{2},\cdots,X_{n})$ 是 $\theta$ 的一个估计量,若 $E\hat \theta_{n} = \theta + k_{n}$, $D\hat \theta_{n} = \sigma_{n}^{2}$,且 $\lim_{n \to \infty} k_{n} = \lim_{n \to \infty} \sigma_{n}^{2} = 0$. 试证:$\hat \theta_{n}$ 是 $\theta$ 的相合(一致)估计量.
	\end{titwo}

	\begin{titwo}
		设总体 $X \sim N\bigl( \mu,\sigma^{2} \bigr)$, $X_{1}$, $X_{2}$, $X_{3}$ 是来自 $X$ 的简单随机样本,试证估计量
		\begin{align*}
			\hat \mu_{1} &= \frac{1}{5} X_{1} + \frac{3}{10} X_{2} + \frac{1}{2} X_{3}, \\
			\hat \mu_{2} &= \frac{1}{3} X_{1} + \frac{1}{4} X_{2} + \frac{5}{12} X_{3}, \\
			\hat \mu_{3} &= \frac{1}{3} X_{1} + \frac{1}{6} X_{2} + \frac{1}{2} X_{3}
		\end{align*}
		都是 $\mu$ 的无偏估计,并指出它们中哪一个最有效.
	\end{titwo}

	\begin{titwo}
		设 $X_{1}$, $X_{2}$, $\cdots$, $X_{n}$ 为来自总体 $X$ 的一个简单随机样本,设 $EX = \mu$, $DX = \sigma^{2}$,试确定常数 $C$,使 $\overline{X}^{2} - CS^{2}$ 为 $\mu^{2}$ 的无偏估计.
	\end{titwo}

	\begin{titwo}
		设总体 $X$ 服从 $U(0,\theta)$, $X_{1}$, $X_{2}$, $\cdots$, $X_{n}$ 为来自总体的简单随机样本. 证明:$\hat \theta = 2 \overline{X}$ 为 $\theta$ 的一致估计.
	\end{titwo}

	\begin{titwo}
		设总体 $X \sim N \bigl( \mu,\sigma^{2} \bigr)$, $X_{1}$, $X_{2}$, $\cdots$, $X_{2n}$ $(n \geq 2)$ 是 $X$ 的简单随机样本,且 $\overline{X} = \frac{1}{2n} \* \sum_{i=1}^{2n} X_{i}$ 及统计量 $Y = \sum_{i=1}^{n} \bigl( X_{i} + X_{n+i} - 2 \overline{X} \bigr)^{2}$.
		\begin{enumerate}
			\item 求统计量 $Y$ 是否为 $\sigma^{2}$ 的无偏估计;
			\item 当 $\mu = 0$ 时,试求 $D \Bigl( \overline{X}^{2} \Bigr)$.
		\end{enumerate}
	\end{titwo}
\section{区间估计与假设检验}
	\begin{titwo}
		设总体 $X \sim N \bigl( \mu,\sigma^{2} \bigr)$,来自 $X$ 的一个样本为 $X_{1}$, $X_{2}$, $\cdots$, $X_{2n}$,记 $\overline{X} = \frac{1}{n} \sum_{i=1}^{n} X_{i}$, $T = \sum_{i=1}^{n} \bigl( X_{i} - \overline{X} \bigr)^{2} + \sum_{i=n+1}^{2n} (X_{i} - \mu)^{2}$,当 $\mu$ 已知时,基于 $T$ 构造估计 $\sigma^{2}$ 的置信水平为 $1 - \alpha$ 的置信区间为 \kuo.

		\twoch{$\Biggl( \frac{T}{\chi_{1 - \frac{\alpha}{2}}^{2}(2n)}, \frac{T}{\chi_{\frac{\alpha}{2}}^{2} (2n) } \Biggr)$}%
		{$\Biggl( \frac{T}{\chi_{\frac{\alpha}{2}}^{2}(2n)}, \frac{T}{\chi_{1 - \frac{\alpha}{2}}^{2} (2n) } \Biggr)$}%
		{$\Biggl( \frac{T}{\chi_{1 - \frac{\alpha}{2}}^{2}(2n-1)}, \frac{T}{\chi_{\frac{\alpha}{2}}^{2} (2n-1) } \Biggr)$}%
		{$\Biggl( \frac{T}{\chi_{\frac{\alpha}{2}}^{2}(2n-1)}, \frac{T}{\chi_{1 - \frac{\alpha}{2}}^{2} (2n-1) } \Biggr)$}
	\end{titwo}

	\begin{titwo}
		设 $\overline{X}$ 为来自总体 $X \sim N \bigl( \mu,\sigma^{2} \bigr)$ ($\sigma^{2}$ 未知)的一个简单随机样本的样本均值,若已知在置信水平 $1 - \alpha$ 下,$\mu$ 的置信区间长度为 $2$,则在显著性水平 $\alpha$ 下,对于假设检验问题 $H_{0}: \mu = 1$, $H_{1}: \mu \ne 1$,要使得检验结果接受 $H_{0}$,则应有 \kuo.

		\twoch{$\overline{X} \in (-1,1)$}{$\overline{X} \in (-1,3)$}{$\overline{X} \in (-2,2)$}{$\overline{X} \in (0,2)$}
	\end{titwo}

	\begin{titwo}
		设总体 $X$ 的方差为 $1$,根据来自 $X$ 的容量为 $100$ 的简单随机样本测得样本均值为 $5$,则 $X$ 的数学期望的置信度近似等于 $0.95$ 的置信区间为 \htwo.
	\end{titwo}

	\begin{titwo}
		设总体 $X \sim N(\mu,8)$, $X_{1}$, $X_{2}$, $\cdots$, $X_{36}$ 是来自 $X$ 的简单随机样本,$\overline{X}$ 是它的均值. 如果 $\bigl( \overline{X} - 1,\overline{X} + 1 \bigr)$ 是未知参数 $\mu$ 的置信区间,则置信水平为 \htwo.
	\end{titwo}

	\begin{titwo}
		设总体 $X \sim N(\mu,8)$, $\mu$ 为未知参数,$X_{1}$, $X_{2}$, $\cdots$, $X_{32}$ 是取自总体 $X$ 的一个简单随机样本,如果以区间 $\bigl[ \overline{X} - 1,\overline{X} + 1 \bigr]$ 作为 $\mu$ 的置信区间,则置信水平为 \htwo. (精确到 $3$ 位小数,参考数值:$\varPhi(2) = 0.977$, $\varPhi(3) \approx 0.999$, $\varPhi(4) \approx 1$)
	\end{titwo}

	\begin{titwo}
		某种零件的尺寸方差为 $\sigma^{2} = 1.21$,抽取一批这类零件中的 $6$ 件检查,得尺寸数据如下(单位:毫米):
		\[
			32.56,29.66,31.64,30.00,21.87,31.03~.
		\]
		设零件尺寸服从正态分布,问这批零件的平均尺寸能否认为是 $32.50$ 毫米 ($\alpha = 0.05$).
	\end{titwo}

	\begin{titwo}
		经测定某批矿砂的 $5$ 个样品中镍含量为 $X$ (\si{\percent}):
		\[
			3.25,3.27,3.24,3.26,3.24,
		\]
		设测定值服从正态分布,问能否认为这批矿砂的镍含量为 $3.25$ ($\alpha = 0.01$)?
	\end{titwo}

	\begin{titwo}
		从一批轴料中取 $15$ 件测量其椭圆度,计算得 $S = 0.025$,问该批轴料椭圆度的总体方差与规定的 $\sigma^{2} = 0.0004$ 有无显著差别. ($\alpha = 0.05$,椭圆度服从正态分布)
	\end{titwo}

	\begin{titwo}
		设某产品的指标服从正态分布,它的标准差为 $\sigma = 100$,今抽了一个容量为 $26$ 的样本,计算平均值为 $1580$,问在显著性水平 $\alpha = 0.05$ 下,能否认为这批产品的指标的期望值 $\mu$ 不低于 $1600$.
    \end{titwo}
    \guanggao
	
	\section{统计量}
	\begin{titwo}
		设 $X_{1}$, $X_{2}$, $\cdots$, $X_{n}$ $(n > 1)$ 是来自总体 $N(0,1)$ 的简单随机样本,记 $\overline{X} = \frac{1}{n} \sum_{i=1}^{n} X_{i}$, $Q^{2} = \sum_{i=1}^{n} X_{i}^{2}$,则 \kuo.

		\onech{$\overline{X} \sim N(0,1)$, $Q^{2} \sim \chi^{2}(n)$}%
		{$\overline{X} \sim N(0,n)$, $Q^{2} \sim \chi^{2}(n-1)$}%
		{$\overline{X} \sim N\bigl(0,\frac{1}{n}\bigr)$, $Q^{2} \sim \chi^{2}(n)$}%
		{$\overline{X} \sim N\bigl(0,\frac{1}{n}\bigr)$, $Q^{2} \sim \chi^{2}(n-1)$}
	\end{titwo}

	\begin{titwo}
		设 $X_{1}$, $X_{2}$, $\cdots$, $X_{8}$ 是来自总体 $N(2,1)$ 的简单随机样本,则统计量
		\[
			Y = \frac{2(X_{1} + X_{2} + X_{3} - 6)}{\sqrt{ 3(X_{4} + X_{5} - 4)^{2} + 2(X_{6} + X_{7} + X_{8} - 6)^{2} }}
		\]
		服从 \kuo.

		\fourch{$\chi^{2}(2)$}{$\chi^{2}(3)$}{$t(2)$}{$t(3)$}
	\end{titwo}

	\begin{titwo}
		设 $X_{1}$, $X_{2}$, $\cdots$, $X_{n}$ 是来自总体 $X \sim N(0,1)$ 的简单随机样本,则统计量
		\[
			Y = \frac{ \sqrt{n - 1}X_{1} }{\sqrt{ \sum_{i=2}^{n} X_{i}^{2} }}
		\]
		服从 \kuo.

		\twoch{$Y \sim \chi^{2}(n-1)$}{$Y \sim t(n-1)$}{$Y \sim F(n,1)$}{$Y \sim F(1,n-1)$}
	\end{titwo}
\end{document}