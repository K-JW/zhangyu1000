\subsection{一元积分的复杂与特色计算}

	\begin{ti}
		设 $I_{n} = \int_{0}^{\frac{\uppi}{4}} \tan^{n}x \dd{x}$($n$ 为非负整数),证明:
		\begin{enumerate}
			\item $I_{n} + I_{n-2} = \frac{1}{n - 1} (n \geq 2)$,并由此求 $I_{n}$;
			\item $\frac{1}{2(n + 1)} < I_{n} < \frac{1}{2(n - 1)}$.
		\end{enumerate}
	\end{ti}

	\begin{ti}
		求 $I_{n} = \int_{-1}^{1} \bigl( x^{2} - 1 \bigr)^{n} \dd{x}$.
	\end{ti}

	\begin{ti}
		设 $a_{n} = \int_{0}^{1} x^{n} \sqrt{1 - x^{2}} \dd{x}$,$b_{n} = \int_{0}^{\frac{\uppi}{2}} \sin^{n}t \dd{t}$,则极限 $\lim_{n \to \infty} \frac{n a_{n}}{b_{n}} = $\kuo.

		\fourch{$1$}{$0$}{$-1$}{$\infty$}
	\end{ti}

	\begin{ti}
		求 $\int_{\ee^{-2 n \uppi}}^{1} \Bigl| \bigl[ \cos \bigl( \ln \frac{1}{x} \bigr) \bigr]' \Bigr| \ln \frac{1}{x} \dd{x}$($n$ 为正整数).
	\end{ti}

	\begin{ti}
		设 $a_{n} = \frac{3}{2} \int_{0}^{\frac{n}{n + 1}} x^{n-1} \sqrt{1 + x^{n}} \dd{x}$,则 $\lim_{n \to \infty} n a_{n} = $ \htwo.
	\end{ti}

	\begin{ti}
		设 $n$ 为正整数,$I_{n} = \int_{0}^{\frac{\uppi}{2}} \frac{\sin 2nx}{\sin x} \dd{x}$.
		\begin{enumerate}
			\item 证明 $I_{n} - I_{n-1} = (-1)^{n-1} \cdot \frac{2}{2n-1} (n \geq 2)$;
			\item 求 $\int_{0}^{\frac{\uppi}{2}} \frac{\sin 6x}{\sin x} \dd{x}$.
		\end{enumerate}
	\end{ti}

	\begin{ti}
		设 $I_{n} = \int_{0}^{\frac{\uppi}{2}} \frac{\sin^{2}nt}{\sin t} \dd{t}$,其中 $n$ 为正整数.
		\begin{enumerate}
			\item 证明 $I_{n} - I_{n-1} = \frac{1}{2n-1}$,并求 $I_{n}$;
			\item 记 $x_{n} = 2I_{n} - \ln n$,证明 $\lim_{n \to \infty} x_{n}$ 存在.
		\end{enumerate}
	\end{ti}

	\begin{ti}
		设函数 $f(x) = x - [x]$,其中 $[x]$ 表示不超过 $x$ 的最大整数,求极限 $\lim_{x \to +\infty} \frac{1}{x} \int_{0}^{x} f(t) \dd{t}$.
	\end{ti}

	\begin{ti}
		设 $x \geq 0$,记 $x$ 到 $2k$ 的最小距离为 $f(x), k = 0,1,2,\cdots$.
		\begin{enumerate}
			\item 证明 $f(x)$ 以 $2$ 为周期;
			\item 求 $\int_{0}^{1} f(nx) \dd{x}$ 的值($n = 1,2,\cdots$).
		\end{enumerate}
	\end{ti}